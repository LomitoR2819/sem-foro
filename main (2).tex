\documentclass{article}
\usepackage{graphicx} % Required for inserting images
\usepackage[spanish]{babel}
\usepackage{xcolor}
\usepackage{enumitem}
\usepackage{tikz}
\usetikzlibrary{shapes, arrows, positioning, calc}
\usepackage[utf8]{inputenc}
\usepackage{enumitem}
\usepackage[margin=1in]{geometry}
\title{practica4}
\author{Abraham Quintero\\ Jesus Contreras}
\date{July 2025}

\begin{document}

\maketitle
\section{}
\begin{figure}[h]
\centering
\begin{tikzpicture}[
    etapa/.style={rectangle, draw=blue!50!black, thick, fill=blue!10, 
                 minimum width=3cm, minimum height=1.5cm, text centered, 
                 rounded corners, font=\small\sffamily},
    flecha/.style={->, >=stealth, thick, blue!50!black},
    titulo/.style={text width=12cm, text centered, font=\large\bfseries\sffamily},
    explicacion/.style={text width=10cm, align=left, font=\small\sffamily}
]

% Título
\node[titulo] at (0, 5) {Ciclo de Atención a una Interrupción de Hardware};

% Nodos del proceso principal
\node[etapa] (evento) at (-5, 3) {1. Evento Externo \\ (Hardware)};
\node[etapa] (senal) at (0, 3) {2. Señal de \\ Interrupción al CPU};
\node[etapa] (fin) at (5, 3) {3. Fin de \\ Instrucción Actual};

% Nodos del proceso de atención
\node[etapa] (salto) at (5, 0) {4. Salto al \\ Manejador (ISR)};
\node[etapa] (ejecucion) at (0, 0) {5. Ejecución \\ de la Rutina};
\node[etapa] (retorno) at (-5, 0) {6. Retorno al \\ Programa};

% Flechas del flujo principal
\draw[flecha] (evento) -- (senal);
\draw[flecha] (senal) -- (fin);
\draw[flecha] (fin) -- (salto);
\draw[flecha] (salto) -- (ejecucion);
\draw[flecha] (ejecucion) -- (retorno);

% Flecha de retorno circular
\draw[flecha] (retorno.south) .. controls +(-1,-1) and +(-1,1) .. (evento.west);

% Línea divisoria
\draw[dashed, gray!50, thick] (-6.5,-1.5) -- (6.5,-1.5);

% Explicación
\node[explicacion, anchor=north west] at (-6.5,-2) {
\textbf{Explicación detallada del ciclo:}\\
\begin{itemize}
    \item[1.] Ocurre un evento físico (dispositivo, timer, etc.)
    \item[2.] El hardware genera señal de interrupción (IRQ)
    \item[3.] CPU termina instrucción actual y guarda estado (PC, registros)
    \item[4.] Salto a ISR mediante vector de interrupciones
    \item[5.] Ejecución del manejador (atención al dispositivo)
    \item[6.] Restauración del estado y retorno al programa original
\end{itemize}
};

% Etiquetas de las fases
\node[blue!50!black, font=\small\sffamily] at (0, 4.2) {\textbf{Generación de la Interrupción}};
\node[blue!50!black, font=\small\sffamily] at (0, -0.8) {\textbf{Atención de la Interrupción}};

\end{tikzpicture}
\caption{Diagrama del ciclo completo de atención a interrupciones hardware}
\end{figure}
\section{Comparación entre Gestión de E/S por Sondeo e Interrupciones}
\subsection{Método de Sondeo (Polling)}

\textbf{Funcionamiento:}
El CPU verifica constantemente el estado de los dispositivos mediante un bucle activo. En cada iteración, comprueba si el dispositivo necesita atención, consumiendo ciclos de procesador continuamente.

\textbf{Ventajas:}
\begin{itemize}[leftmargin=*]
    \item Implementación simple y directa
    \item Comportamiento determinista (ideal para sistemas de tiempo real críticos)
    \item No requiere hardware adicional
    \item Fácil de depurar y monitorear
\end{itemize}

\textbf{Desventajas:}
\begin{itemize}[leftmargin=*]
    \item Ineficiencia energética (CPU siempre activo)
    \item Desperdicio significativo de ciclos de procesamiento
    \item Latencia dependiente de la frecuencia de verificación
    \item Escalabilidad limitada con múltiples dispositivos
\end{itemize}

\textbf{Ejemplo típico en código:}
El CPU ejecuta un bucle que verifica un registro de estado del dispositivo, continuando con el procesamiento solo cuando detecta que los datos están listos.

\subsection{Método por Interrupciones}

\textbf{Funcionamiento:}
Los dispositivos generan señales de interrupción cuando requieren atención. El CPU suspende su actividad actual, salva el contexto, ejecuta una rutina de servicio (ISR) específica para atender el dispositivo, y luego restaura el contexto para continuar.

\textbf{Ventajas:}
\begin{itemize}[leftmargin=*]
    \item Máxima eficiencia del CPU (activo solo cuando es necesario)
    \item Bajo consumo energético
    \item Respuesta inmediata a eventos
    \item Escalabilidad óptima para múltiples dispositivos
    \item Priorización flexible de eventos
\end{itemize}

\textbf{Desventajas:}
\begin{itemize}[leftmargin=*]
    \item Mayor complejidad de implementación
    \item Overhead por gestión de contexto (salvar/restaurar registros)
    \item Requiere hardware especializado (controlador de interrupciones)
    \item Comportamiento menos predecible en términos temporales
\end{itemize}

\textbf{Ejemplo típico:}
Cuando un dispositivo como un teclado tiene datos disponibles, interrumpe al CPU, que ejecuta una rutina específica para leer el buffer de entrada antes de retomar la tarea previa.

\subsection{Diferencias Clave}

\textbf{Uso del CPU:}
El sondeo mantiene al CPU ocupado verificando estados incluso cuando no hay trabajo real que hacer, mientras que las interrupciones permiten que el procesador se dedique a otras tareas hasta que sea realmente necesario atender un dispositivo.

\textbf{Latencia:}
En el sondeo, el tiempo de respuesta depende directamente de la frecuencia con la que se verifica el dispositivo. Con interrupciones, la latencia está determinada principalmente por el hardware y es generalmente más consistente.

\textbf{Implementación:}
El sondeo puede implementarse con código mínimo y sin soporte hardware especial. Las interrupciones requieren mecanismos para salvar/restaurar contexto, vectores de interrupción y rutinas de servicio bien definidas.

\textbf{Escalabilidad:}
Mientras que agregar más dispositivos en un sistema por sondeo ralentiza proporcionalmente el rendimiento, las interrupciones permiten manejar múltiples dispositivos eficientemente mediante prioridades y manejo adecuado de colas.

\subsection{Recomendaciones de Uso}

\textbf{Use sondeo cuando:}
\begin{itemize}[leftmargin=*]
    \item Los tiempos de respuesta del dispositivo son extremadamente rápidos
    \item Se requiere comportamiento completamente predecible
    \item Los recursos hardware son muy limitados
    \item La simplicidad de implementación es prioritaria
\end{itemize}

\textbf{Prefiera interrupciones cuando:}
\begin{itemize}[leftmargin=*]
    \item Los dispositivos tienen tiempos de respuesta variables o largos
    \item La eficiencia energética es importante
    \item El sistema debe manejar múltiples dispositivos concurrentemente
    \item Se necesita máxima utilización del CPU para otras tareas
\end{itemize}
\section{Ventajas del Uso de Interrupciones}

Las interrupciones ofrecen importantes beneficios en la gestión eficiente del procesador:

\begin{itemize}[leftmargin=*]
    \item \textbf{Maximización del tiempo útil:}
    \begin{itemize}
        \item El procesador puede ejecutar otras tareas mientras espera eventos de E/S
        \item No consume ciclos verificando dispositivos innecesariamente
        \item Aprovechamiento óptimo de la capacidad de cálculo
    \end{itemize}
    
    \item \textbf{Reducción del consumo energético:}
    \begin{itemize}
        \item Permite poner el procesador en estados de bajo consumo durante esperas
        \item Elimina el gasto energético de bucles de sondeo activos
        \item Ideal para sistemas con restricciones de energía (móviles, embebidos)
    \end{itemize}
    
    \item \textbf{Respuesta inmediata a eventos:}
    \begin{itemize}
        \item Atención en tiempo real a dispositivos cuando lo requieren
        \item Latencia predecible determinada por el hardware
        \item No depende de frecuencias de muestreo como en el sondeo
    \end{itemize}
    
    \item \textbf{Escalabilidad con múltiples dispositivos:}
    \begin{itemize}
        \item Manejo eficiente de numerosos periféricos simultáneamente
        \item Priorización inteligente mediante controlador de interrupciones
        \item El costo computacional no aumenta linealmente con más dispositivos
    \end{itemize}
    
    \item \textbf{Arquitectura más limpia:}
    \begin{itemize}
        \item Separación clara entre código principal y rutinas de servicio
        \item Mejor organización del flujo del programa
        \item Más fácil de mantener y ampliar funcionalidades
    \end{itemize}
\end{itemize}

\subsection{Impacto en el Rendimiento}

El uso de interrupciones mejora significativamente el rendimiento del sistema porque:

\begin{itemize}[leftmargin=*]
    \item Elimina el \textit{overhead} por verificación constante de estados
    \item Reduce los conflictos por acceso a buses de E/S
    \item Permite paralelismo real entre cómputo y operaciones de E/S
    \item Facilita la implementación de sistemas multitarea
\end{itemize}
\section{Registros para Gestión de Interrupciones en MIPS32}
\subsection{Registros del Coprocesador 0 (CP0)}

El conjunto de instrucciones MIPS32 utiliza registros especiales del Coprocesador 0 para gestionar interrupciones. Los principales son:

\subsection{Status Register (SR - \$12)}
Registro de estado con los siguientes campos relevantes:
\begin{itemize}
\item IE (bit 0): Interrupt Enable - Habilita globalmente las interrupciones cuando está a 1
\item EXL (bit 1): Exception Level - Indica que se está ejecutando una excepción cuando está a 1
\item IM (bits 15-8): Interrupt Mask - Habilita interrupciones específicas (1=habilitada)
\item KSU (bits 4-3): Kernel/Supervisor/User Mode - Indica el nivel de privilegio actual
\end{itemize}

\subsection{Cause Register (\$13)}
Registro de causa con los siguientes campos:
\begin{itemize}
\item IP (bits 15-8): Interrupt Pending - Muestra qué interrupciones están pendientes
\item ExcCode (bits 6-2): Exception Code - Identifica el tipo de excepción (6 para interrupción hardware)
\end{itemize}

\subsection{EPC Register (\$14)}
Registro que almacena el Program Counter (PC) en el momento de ocurrir la interrupción, permitiendo retornar a la instrucción correcta después de atenderla.

\subsection{Config Register (\$16)}
Registro de configuración del sistema que puede afectar el manejo de interrupciones, aunque su uso principal es para configuración general del procesador.

\subsection{Ejemplo de Configuración}

Para habilitar interrupciones en MIPS32 se debe:

1. Configurar los bits IM en el registro Status para habilitar las interrupciones deseadas
2. Poner a 1 el bit IE en el registro Status para habilitar globalmente las interrupciones
3. Configurar el vector de interrupciones en memoria
4. Implementar las rutinas de servicio de interrupción (ISR)

\subsection{Flujo de Atención de Interrupciones}

Cuando ocurre una interrupción:

1. El procesador completa la instrucción actual en ejecución
2. Guarda el PC en el registro EPC
3. Pone a 1 el bit EXL en el registro Status
4. Salta a la dirección del vector de interrupciones
5. La rutina ISR debe:
   - Identificar la fuente de interrupción (leyendo Cause)
   - Atender el dispositivo que generó la interrupción
   - Limpiar la condición de interrupción
   - Usar la instrucción ERET para retornar

La instrucción ERET restaura el estado anterior poniendo EXL a 0 y retomando la ejecución en la dirección almacenada en EPC.
\section{Preservación del Contexto en Rutinas de Interrupción}
\subsection{Necesidad de Guardar el Contexto}

Al entrar en una Rutina de Servicio de Interrupción (ISR), es \textbf{imprescindible} guardar el contexto del programa interrumpido por las siguientes razones técnicas:

\subsection{Integridad del Estado del Procesador}
\begin{itemize}
\item Los registros del CPU contienen valores críticos del programa interrumpido
\item Cualquier modificación durante la ISR corrompería el estado original
\item Sin preservación, al retornar el programa se comportaría erráticamente
\end{itemize}

\subsection{Transparencia de la Interrupción}
\begin{itemize}
\item Las interrupciones deben ser transparentes al programa principal
\item El programa no debe detectar que fue interrumpido (excepto por el retardo)
\item Requiere que todos los registros mantengan sus valores originales
\end{itemize}

\subsection{Registros Críticos a Preservar}
\begin{itemize}
\item \textbf{Registros de propósito general} (\$t0-\$t9, \$a0-\$a3, etc.)
\item \textbf{Registro de estado} (SR) - Contiene flags importantes
\item \textbf{Contador de programa} (PC) - Guardado automáticamente en EPC
\item \textbf{Registros de coprocesador} - Si son utilizados por el programa
\end{itemize}

\subsection{Mecanismos de Preservación}

\subsection{Salvado Manual}
En arquitecturas como MIPS32, el programador debe guardar explícitamente:
\begin{itemize}
\item Los registros que la ISR modificará
\item El registro \$ra si usa llamadas a subrutinas
\item Flags importantes del registro de estado
\end{itemize}

\subsection{Ejemplo en MIPS32}
\begin{verbatim}
ISR:
    # Guardar contexto
    sw $t0, save_t0
    sw $t1, save_t1
    mfc0 $k0, $12       # Leer Status
    sw $k0, save_status
    
    # Cuerpo de la ISR
    ...
    
    # Restaurar contexto
    lw $k0, save_status
    mtc0 $k0, $12       # Escribir Status
    lw $t0, save_t0
    lw $t1, save_t1
    eret
\end{verbatim}

\subsection{Consecuencias de No Preservar}
\begin{itemize}
\item \textbf{Corrupción de datos}: Valores de registros se pierden
\item \textbf{Fallos aleatorios}: Comportamiento no determinista
\item \textbf{Problemas de reinicio}: El programa no puede continuar correctamente
\item \textbf{Dificultad de depuración}: Errores difíciles de rastrear
\end{itemize}

\subsection{Consideraciones de Rendimiento}
\begin{itemize}
\item El salvado/restauración añade overhead (15-30 ciclos típicos)
\item Se optimiza guardando solo los registros necesarios
\item En sistemas críticos se usan registros dedicados para ISRs
\item Arquitecturas modernas implementan salvado automático parcial
\end{itemize}
\section{Excepciones en Arquitectura MIPS32}
\subsection{Situaciones Generadoras de Excepciones}

En MIPS32, las excepciones pueden generarse en las siguientes situaciones:

\begin{enumerate}
    \item \textbf{Desbordamiento aritmético (Overflow)}:
    \begin{itemize}
        \item Ocurre en operaciones como \texttt{add}, \texttt{sub} cuando el resultado excede el rango representable
        \item Ejemplo: $2^{31} + 2^{31}$ en enteros de 32 bits con signo
    \end{itemize}

    \item \textbf{Fallo de dirección (Address Error)}:
    \begin{itemize}
        \item Acceso a direcciones no alineadas (ej. \texttt{lw} en dirección no múltiplo de 4)
        \item Acceso a memoria fuera del espacio direccionable
    \end{itemize}

    \item \textbf{Instrucción no implementada (Reserved Instruction)}:
    \begin{itemize}
        \item Intento de ejecutar un código de operación no definido
        \item Ejecución de instrucción privilegiada en modo usuario
    \end{itemize}

    \item \textbf{Interrupción externa (Hardware Interrupt)}:
    \begin{itemize}
        \item Señal de dispositivos periféricos (timer, UART, etc.)
        \item Requiere configuración previa en registros \texttt{Status} y \texttt{Cause}
    \end{itemize}

    \item \textbf{Fallo de página TLB (TLB Miss)}:
    \begin{itemize}
        \item Cuando la traducción de dirección virtual a física falla
        \item Ocurre tanto en carga como en almacenamiento
    \end{itemize}

    \item \textbf{Breakpoint/Syscall}:
    \begin{itemize}
        \item Generadas deliberadamente por instrucciones \texttt{syscall} y \texttt{break}
        \item Mecanismo para llamadas al sistema y depuración
    \end{itemize}
\end{enumerate}

\subsection{Etapas del Pipeline y Excepciones}

El pipeline de 5 etapas en MIPS32 puede generar excepciones en diferentes fases:

\begin{enumerate}
    \item \textbf{IF (Instruction Fetch)}:
    \begin{itemize}
        \item \texttt{TLB Miss} en acceso a memoria de instrucciones
        \item \texttt{Bus Error} por fallo en lectura
        \item \texttt{Address Error} por dirección no alineada
    \end{itemize}

    \item \textbf{ID (Instruction Decode)}:
    \begin{itemize}
        \item \texttt{Reserved Instruction} al decodificar opcode inválido
        \item \texttt{Coprocessor Unavailable} para instrucciones de CP no habilitado
    \end{itemize}

    \item \textbf{EX (Execute)}:
    \begin{itemize}
        \item \texttt{Overflow} en operaciones aritméticas
        \item \texttt{Integer Divide by Zero} en división
        \item \texttt{Trap} en comparaciones condicionales
    \end{itemize}

    \item \textbf{MEM (Memory Access)}:
    \begin{itemize}
        \item \texttt{TLB Miss} en acceso a datos
        \item \texttt{Bus Error} en escritura/lectura
        \item \texttt{Address Error} por desalineamiento
    \end{itemize}

    \item \textbf{WB (Write Back)}:
    \begin{itemize}
        \item Normalmente no genera excepciones
        \item Puede detectar \texttt{TLB Miss} retardado en algunas implementaciones
    \end{itemize}
\end{enumerate}

\subsection{Jerarquía de Manejo}

Cuando múltiples excepciones ocurren simultáneamente en diferentes etapas del pipeline, MIPS32 prioriza según:

\begin{enumerate}
    \item Excepciones de etapas tempranas (IF > ID > EX > MEM)
    \item Dentro de la misma etapa, prioridad fija por tipo
    \item El bit \texttt{EXL} en \texttt{Status} evite anidamiento no deseado
\end{enumerate}
\subsection{Diferencias entre Interrupciones y Excepciones}

\begin{description}
\item[Interrupciones:]
\begin{itemize}
    \item \textbf{Asíncronas}: Ocurren independientemente del flujo de ejecución del programa
    \item Generadas por eventos externos al procesador (periféricos, timer, etc.)
    \item Pueden habilitarse/deshabilitarse mediante el registro \texttt{Status}
    \item Ejemplo: Llegada de datos por UART, fin de temporización
\end{itemize}

\item[Excepciones:]
\begin{itemize}
    \item \textbf{Síncronas}: Relacionadas directamente con la ejecución de instrucciones
    \item Generadas por condiciones especiales durante la ejecución
    \item No pueden deshabilitarse (siempre activas)
    \item Ejemplo: Overflow, acceso inválido a memoria, instrucción no definida
\end{itemize}
\end{description}
\section{Estrategias de Tratamiento de Excepciones e Interrupciones en MIPS32}
\subsection{Estrategias para Tratar Excepciones en MIPS32}

\subsection{1. Manejo mediante Vectores Fijos}
\begin{itemize}
    \item Todas las excepciones saltan a una dirección fija: \texttt{0x80000180} (kseg1)
    \item La rutina principal lee el registro \texttt{Cause} para determinar la causa exacta
    \item Ventaja: Simple y predecible
    \item Desventaja: Requiere análisis secuencial de causas
\end{itemize}

\textbf{Flujo:}
\begin{enumerate}
    \item El hardware guarda automáticamente el PC en \texttt{EPC}
    \item Salta a la dirección del vector fijo
    \item Software consulta \texttt{Cause} y \texttt{BadVAddr} si aplica
    \item Ejecuta la rutina específica
    \item Retorna con \texttt{eret} (restaura PC desde \texttt{EPC})
\end{enumerate}

\subsection{2. Manejo mediante Tabla de Vectores}
\begin{itemize}
    \item Implementación común en versiones posteriores (MIPS64)
    \item Diferentes causas saltan a direcciones distintas
    \item Ventaja: Menor latencia en el manejo
    \item Desventaja: Mayor complejidad hardware
\end{itemize}

\textbf{Flujo:}
\begin{enumerate}
    \item Hardware determina el vector según \texttt{ExcCode}
    \item Salta a la dirección correspondiente en la tabla
    \item No necesita análisis software inicial
    \item Retorno igual con \texttt{eret}
\end{enumerate}

\subsection{Función del Registro EPC}
El \textbf{Exception Program Counter (EPC)} cumple roles críticos:
\begin{itemize}
    \item Almacena la dirección de la instrucción que causó la excepción
    \item En interrupciones, guarda la dirección de la \textbf{siguiente} instrucción a ejecutar
    \item Permite reanudar la ejecución normal después del manejo
    \item En excepciones recuperables (ej. TLB Miss), permite reintentar la instrucción
    \item Su valor se restaura automáticamente con \texttt{eret}
\end{itemize}

\subsection{Ejemplo de Código}
\begin{verbatim}
.ktext 0x80000180    # Vector fijo
    mfc0 $k0, $13    # Leer Cause
    srl $k0, 2       # Aislar ExcCode
    andi $k0, 0x1f
    # Saltar a rutina según valor
    eret             # Retornar
\end{verbatim}
\section{Habilitación de Interrupciones en Sistema MIPS32}
\subsection{Habilitación de Interrupciones por Dispositivo}

\subsection{Teclado}
\begin{itemize}
\item Se configura mediante el registro de control del controlador de teclado (ej. PS/2 o USB)
\item Habilitación típica:
  \begin{enumerate}
  \item Escribir `1' en el bit de habilitación de interrupciones del registro de control del teclado
  \item Configurar el vector de interrupción correspondiente en el controlador de interrupciones
  \item Establecer prioridad si el controlador lo soporta
  \end{enumerate}
\item Ejemplo código:
\begin{verbatim}
li $t0, 0xFFFF0000   # Dirección base teclado
li $t1, 0x00000002   # Bit 1 = Habilitar interrupción
sw $t1, 0($t0)       # Escribir en registro de control
\end{verbatim}
\end{itemize}

\subsection{Pantalla}
\begin{itemize}
\item Depende del controlador gráfico (ej. framebuffer)
\item Pasos típicos:
  \begin{enumerate}
  \item Habilitar interrupción de refresco vertical (VSync)
  \item Configurar dirección ISR en el controlador
  \item Establecer máscara de interrupción
  \end{enumerate}
\item Ejemplo:
\begin{verbatim}
li $t0, 0xFFFF8000   # Dirección controlador video
li $t1, 0x00000001   # Habilitar VSync interrupt
sw $t1, 4($t0)       # Escribir en registro de control
\end{verbatim}
\end{itemize}

\subsection{Habilitación en el Procesador}

En MIPS32 se configuran los registros del Coprocesador 0:

\begin{itemize}
\item \textbf{Status Register (\$12)}:
  \begin{itemize}
  \item Bit 0 (IE): Interrupt Enable global (1=habilitado)
  \item Bits 15-8 (IM): Máscara de interrupciones hardware (1=habilitar)
  \item Bit 1 (EXL): Debe ser 0 para permitir interrupciones
  \end{itemize}

\item \textbf{Cause Register (\$13)}:
  \begin{itemize}
  \item Bits 15-8 (IP): Muestra interrupciones pendientes
  \end{itemize}

\item \textbf{Ejemplo de habilitación}:
\begin{verbatim}
mfc0 $t0, $12        # Leer Status
ori $t0, 0x00000101   # IE=1, IM[0]=1 (habilitar int. 0)
mtc0 $t0, $12        # Escribir Status
\end{verbatim}
\end{itemize}

\subsection{Consecuencias de Habilitar Solo en Dispositivos}

Si los dispositivos generan interrupciones pero el procesador no está configurado para atenderlas:

\begin{itemize}
\item \textbf{Señales perdidas}: Las interrupciones quedan pendientes en el registro \texttt{IP} (Cause)
\item \textbf{Comportamiento indefinido}: Algunos controladores pueden saturar el bus
\item \textbf{Contaminación de estado}: Los bits pendientes pueden afectar futuras interrupciones
\item \textbf{Posible deadlock}: Dispositivos que esperan atención pueden bloquearse
\end{itemize}

\textbf{Solución}: Siempre seguir la secuencia:
\begin{enumerate}
\item Configurar manejadores (ISR)
\item Habilitar interrupciones en dispositivos
\item Finalmente habilitar globalmente en el procesador
\end{enumerate}
\section{Procesamiento de Interrupciones en MIPS32}
\subsection{Proceso de Atención a Interrupción de Reloj}

\subsection{Paso a Paso}
\begin{enumerate}
    \item \textbf{Generación de la interrupción}:
    \begin{itemize}
        \item El temporizador hardware alcanza el valor programado
        \item El dispositivo de reloj activa la línea IRQ correspondiente
    \end{itemize}

    \item \textbf{Reconocimiento por el procesador}:
    \begin{itemize}
        \item El CPU completa la instrucción actual en ejecución
        \item Establece el bit \texttt{IP[7]} en el registro \texttt{Cause}
        \item Verifica que \texttt{IE=1} y \texttt{EXL=0} en \texttt{Status}
    \end{itemize}

    \item \textbf{Acciones del hardware}:
    \begin{itemize}
        \item Guarda automáticamente el PC en \texttt{EPC}
        \item Establece \texttt{EXL=1} (nivel excepción)
        \item Salta a la dirección del vector de interrupciones (\texttt{0x80000180} en MIPS32)
    \end{itemize}

    \item \textbf{Rutina de servicio (software)}:
    \begin{itemize}
        \item Guarda manualmente los registros que usará:
\begin{verbatim}
sw $at, 0($sp)
sw $v0, 4($sp)
sw $a0, 8($sp)
# ... etc
\end{verbatim}
        \item Lee el registro \texttt{Cause} para confirmar la fuente
        \item Atiende el dispositivo (reinicia el temporizador)
        \item Restaura los registros guardados
        \item Ejecuta \texttt{eret} (instrucción de retorno de excepción)
    \end{itemize}

    \item \textbf{Retorno (hardware)}:
    \begin{itemize}
        \item Restaura \texttt{PC} desde \texttt{EPC}
        \item Limpia \texttt{EXL} (\texttt{EXL=0})
        \item Reanuda ejecución del programa interrumpido
    \end{itemize}
\end{enumerate}

\subsection{Registros Involucrados}
\begin{itemize}
    \item \textbf{Guardados automáticamente}:
    \begin{itemize}
        \item \texttt{EPC}: Contador de programa
        \item \texttt{SR}: Registro de estado (solo \texttt{EXL} se modifica)
    \end{itemize}

    \item \textbf{Guardados manualmente}:
    \begin{itemize}
        \item Registros de propósito general (\texttt{\$at, \$v0, \$a0, etc.})
        \item Registros del coprocesador si se usan
        \item Registro \texttt{\$ra} si hay llamadas anidadas
    \end{itemize}

    \item \textbf{Restaurados automáticamente}:
    \begin{itemize}
        \item \texttt{PC} desde \texttt{EPC} (vía \texttt{eret})
        \item \texttt{SR.EXL} se limpia
    \end{itemize}
\end{itemize}

\subsection{Importancia de Guardar el Contexto}

El salvado de contexto es crítico porque:

\begin{itemize}
    \item \textbf{Transparencia}: El programa interrumpido no debe percibir cambios en sus registros
    \item \textbf{Consistencia}: Previene corrupción de datos entre rutinas
    \item \textbf{Estado preciso}: Permite reanudación exacta de la ejecución
    \item \textbf{Anidamiento}: Habilita el manejo de interrupciones durante otras interrupciones
\end{itemize}

\textbf{Consecuencias de no guardar contexto}:
\begin{itemize}
    \item Corrupción de registros usados por el programa principal
    \item Comportamiento errático del sistema
    \item Pérdida de datos críticos
    \item Fallos difíciles de depurar
\end{itemize}

\subsection{Ejemplo Completo}
\begin{verbatim}
.ktext 0x80000180
    # 1. Guardar contexto
    sw $at, 0($sp)
    sw $v0, 4($sp)
    # ... (todos los registros a usar)

    # 2. Atender interrupción
    mfc0 $k0, $13       # Leer Cause
    andi $k0, 0x8000    # Verificar bit 15 (Timer)
    beqz $k0, not_timer
    
    # 3. Reiniciar timer
    li $t0, 0xFFFF0010
    lw $t1, 0($t0)      # Leer contador
    addi $t1, 1000      # Nuevo valor
    sw $t1, 0($t0)      # Escribir

not_timer:
    # 4. Restaurar contexto
    lw $at, 0($sp)
    lw $v0, 4($sp)
    # ...

    # 5. Retornar
    eret
\end{verbatim}
\subsection{Interrupciones de Reloj y Control de Ejecución en MIPS32}
\section{Uso de Interrupciones de Reloj}

\subsection{Prevención de Bucles Infinitos}
El sistema operativo puede configurar el temporizador hardware para generar interrupciones periódicas:

\begin{itemize}
\item \textbf{Configuración inicial}:
\begin{verbatim}
# Configurar intervalo del timer (ej. 10ms)
li $t0, 0xFFFF0010   # Dirección del timer
li $t1, 10000        # Valor inicial
sw $t1, 0($t0)       # Escribir valor
\end{verbatim}

\item \textbf{Mecanismo de protección}:
\begin{enumerate}
\item El SO mantiene un contador por proceso
\item En cada interrupción de reloj:
\begin{verbatim}
subi $s0, $s0, 1     # Decrementar contador
blez $s0, kill_proc  # Si <=0, terminar proceso
\end{verbatim}
\item Si el contador llega a cero, se fuerza la terminación
\end{enumerate}
\end{itemize}

\subsection{Limitación de Tiempo de Ejecución}
Para programas con tiempo máximo definido:

\begin{itemize}
\item \textbf{Implementación típica}:
\begin{enumerate}
\item Asignar quantum de tiempo al cargar el proceso
\item Programar el timer con el quantum
\item En la ISR:
\begin{verbatim}
# Verificar si es el proceso actual
lw $t2, current_pid
bne $t2, $s1, skip
# Forzar cambio de contexto
jal scheduler
skip:
\end{verbatim}
\end{enumerate}
\end{itemize}

\subsection{Manejo de Finalización Anticipada}

Cuando un programa termina antes del intervalo del timer:

\begin{enumerate}
\item \textbf{Liberación de recursos}:
\begin{itemize}
\item El SO debe desprogramar el temporizador:
\begin{verbatim}
sw $zero, 0($t0)    # Detener timer
\end{verbatim}
\item Liberar el contador asociado al proceso
\end{itemize}

\item \textbf{Optimizaciones}:
\begin{itemize}
\item Reutilizar el quantum restante para otros procesos
\item Actualizar estadísticas de uso de CPU
\item Notificar al planificador para inmediata reasignación
\end{itemize}

\item \textbf{Caso especial - Syscall exit}:
\begin{verbatim}
# En handler de syscall:
li $v0, 10         # Código de exit
# ... limpieza ...
sw $zero, 0($t0)   # Desactivar timer
jr $ra             # Retornar al scheduler
\end{verbatim}
\end{enumerate}

\subsection{Consideraciones de Implementación}

\begin{itemize}
\item \textbf{Sincronización}: El SO debe proteger estructuras de datos compartidas con el ISR
\item \textbf{Overhead}: Balancear frecuencia de interrupciones vs. responsividad
\item \textbf{Prioridades}: Asignar adecuadamente la prioridad del timer
\item \textbf{Recuperación}: Manejar adecuadamente el caso donde el ISR es interrumpido
\end{itemize}
\end{document}
